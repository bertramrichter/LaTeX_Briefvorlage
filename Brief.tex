\documentclass[twoside, A]{brbrief}
\usepackage[ngerman]{babel}			% Spracheinstellungen, zwingend benötigt
\usepackage[utf8]{inputenc}			% Wahl der Zeichenencodierung, benötigt
\usepackage[OT1]{fontenc}			% Intelligente Ligaturen, benötigt
\usepackage{CormorantGaramond}		% schicke Schrift, optional
\usepackage{pdfpages}				% PDF-Dateien einfügen, optional
\usepackage[useregional]{datetime2}	% Daten formatieren, optional
\usepackage{lipsum}					% Lorem-ipsum-Text-Generator, kann entfernt werden

% Absenderfeld
\Informationsblock{
	\flushright
	Ihr Zeichen: 123456 \\
	Ihr Schreiben vom 1234-56-78\\
	Kundennummer: 0123456789 \\
	~\\
	Name des Absenders \\
	Absenderstraße 1 \\
	01234 Absenderort \\
	~\\
	\phonesymbol{} Tel: +49~123~123456789 \\
	\mobilesymbol{} Mobil: +49~123~123456789 \\
	\mailsymbol{} Email: \href{mailto:abc@abs.url}{abc@abc.url} \\
	~\\
	Zusätzliche Infos zum Absender \\
	~\\
	\DTMtoday{}
	}
% Adressfeld
\Vermerkzone{%
	elektronische Freimachungsvermerke \\
	Vorausfügung \\
	z.\,B.\@ Einschreiben \\
	{Name des Absenders, Absenderstraße 1, 01234 Absenderort}
	}
\Anschriftzone{%
	Firma \\
	Frau\slash{}Herr \\
	Name des Adressaten \\
	Zielstraße 1 \\
	12345 Zielort \\
	Deutschland \\
	}
\Betreff{Betreff}
\Signatur{Unterschrift.png}

\begin{document}
	\maketitle{}
	\Anrede{Liebe(r) Adressat(in),}
	
	\lipsum[1-3]
	
	\Gruss{mit freundlichem Gruß}{Absendername}
	
	% Post Scriptum
	%PS: Ich bin bis März nur telefonisch erreichbar.
	
	% Weitere PDFs können automatisch angefügt werden, z.B. Anhänge.
	%\includepdf[pages=-,openright]{pfad/zu/weiteren/pdfs/dokument.pdf}
	% Pfad ist relativ zu dieser tex-Datei. Mit .. ein Verzeichnis hoch.
	% Der pages-Parameter spezifiziert welche Seiten eingefügt werden.
	% Beispiele:
	% pages=-				alle Seiten
	% pages={1-4}			Seite 1-4
	% pages={1,4,5}			Seite 1, 4 und 5
	% pages={3,{},8-11,15}	Seite 3, leere Seite, Seite 8-11 und Seite 15
	% Der openright-Parameter startet die Anlagen auf ungerader (rechter)
	% Seite, d. h. notfalls wird eine leere Seite eingefügt.
	% Im doppelseitigem Druck wird dadurch besser zwischen Brief und Anlage
	% getrennt. Für einseitigen Druck entfernen.
\end{document}
