\documentclass[twoside]{brbrief}

\usepackage{lipsum}

% Development section begin
\makeatletter

\makeatother
% Development section end


% Absenderfeld
\ABSName{Name des Absenders}
\ABSStrasse{Absenderstraße 1}
\ABSPLZ{01234}
\ABSOrt{Absenderort}
\ABSPhone{+49~123~123456789}
\ABSMobile{+49~123~123456789}
\ABSMail{absender@provider.url}
\ABSInfo{Zusätzliche Informationen über den Absender}

% Adressfeld
\ZViii{elektronische Freimachungsvermerke}
\ZVii{Vorausfügung}
\ZVi{z.\,B.\ Einschreiben}
\AZFirma{Firma}
\AZAnrede{Frau/Herr}
\AZName{Name des Addressaten}
\AZStrasse{Zielstraße 1}
\AZPLZOrt{12345 Zielort}
\AZLand{Deutschland}

% Betreff
\IhrZeichen{Ihr Zeichen}
\IhrSchreiben{01.23.4567}
\KundenNummer{0123456789}
\Betreff{Betreff}

% Aussehen
\Briefkopf{}
\Briefpapier{}
\Signature{Unterschrift.png}

\begin{document}
	\maketitle{}
	\Anrede{Liebe(r) Adressat(in)}
	
	\lipsum[1-2]
	
	Zu tun:
	\begin{itemize}
		\item Strich in Adressfeld genau so lang, wie längster Eintrag
		\item Dokumentation für jede Option
		\item Location Absenderinfos configurable
		\item Location Absenderinfos configurable
	\end{itemize}
	
	\Gruss{Dein}
	
	% Post Scriptum
	%PS: Ich bin bis März nur telefonisch erreichbar.
	
	% Weitere PDFs können automatisch angefügt werden, z.B. Ahnänge.
	%\includepdf[pages=-,openright]{pfad/zu/weiteren/pdfs/dokument.pdf}
	% Pfad ist relativ zu dieser tex-Datei. Mit .. ein Verzeichnis hoch.
	% Der pages-Parameter spezifiziert welche Seiten eingefügt werden.
	% Beispiele:
	% pages=-				alle Seiten
	% pages={1-4}			Seite 1-4
	% pages={1,4,5}			Seite 1, 4 und 5
	% pages={3,{},8-11,15}	Seite 3, leere Seite, Seite 8-11 und Seite 15
	% Der openright-Parameter startet die Anlagen auf ungerader (rechter) Seite, d.h. notfalls wird eine leere Seite
	% eingefügt. Im doppelseitigem Druck wird dadurch besser zwischen Brief und Anlage getrennt. Für einseitigen Druck
	% entfernen.
	
	
\end{document}
