\documentclass[twoside, A]{brbrief}
\usepackage{lipsum}				% Lorem-ipsum-Text-Generator
\usepackage[ngerman]{babel}		% for hyphenation and date format
\usepackage[utf8]{inputenc}		% 
\usepackage[OT1]{fontenc}		% Intelligente Ligaturen
\usepackage{CormorantGaramond}	% schicke Schrift
\usepackage{pdfpages}

% Absenderfeld
\Informationsblock{
	\flushright
	Ihr Zeichen: 123456 \\
	Ihr Schreiben vom 1234-56-78\\
	Kundennummer: 0123456789 \\
	~\\
	Name des Absenders \\
	Absenderstraße 1 \\
	01234 Absenderort \\
	~\\
	\phonesymbol{} Tel: +49~123~123456789 \\
	\mobilesymbol{} Mobil: +49~123~123456789 \\
	\mailsymbol{} Email: \href{mailto:absender@provider.url}{absender@provider.url} \\
	~\\
	
	Zusätzliche Infos zum Absender
	}
% Adressfeld
\Vermerkzone{%
	elektronische Freimachungsvermerke \\
	Vorausfügung \\
	z.\,B.\@ Einschreiben \\
	Name des Absenders, Absenderstraße 1, 01234 Absenderort%
	}
\Anschriftzone{%
	Firma \\
	Frau\slash{}Herr \\
	Name des Adressaten \\
	Zielstraße 1 \\
	12345 Zielort \\
	Deutschland \\
	}
% Betreff
\Betreff{Betreff}
% Aussehen
\Briefkopf{}
\Signatur{Unterschrift.png}

\begin{document}
	
	\maketitle{}
	\Anrede{Liebe(r) Adressat(in)}
	
	\lipsum[1-10]
	
	Zu tun:
	\begin{itemize}
		\item Strich in Adressfeld genau so lang, wie längster Eintrag
		\item Location Absenderinfos configurable
		\item Briefpapier konfigurabel: Erste Seite, Rechte Seite, Linke Seite
	\end{itemize}
	
	\Gruss{mit freundlichem Gruß}{Absendername}
	
	% Post Scriptum
	%PS: Ich bin bis März nur telefonisch erreichbar.
	
	% Weitere PDFs können automatisch angefügt werden, z.B. Anhänge.
	%\includepdf[pages=-,openright]{pfad/zu/weiteren/pdfs/dokument.pdf}
	% Pfad ist relativ zu dieser tex-Datei. Mit .. ein Verzeichnis hoch.
	% Der pages-Parameter spezifiziert welche Seiten eingefügt werden.
	% Beispiele:
	% pages=-				alle Seiten
	% pages={1-4}			Seite 1-4
	% pages={1,4,5}			Seite 1, 4 und 5
	% pages={3,{},8-11,15}	Seite 3, leere Seite, Seite 8-11 und Seite 15
	% Der openright-Parameter startet die Anlagen auf ungerader (rechter) Seite, d.h. notfalls wird eine leere Seite
	% eingefügt. Im doppelseitigem Druck wird dadurch besser zwischen Brief und Anlage getrennt. Für einseitigen Druck
	% entfernen.
\end{document}
