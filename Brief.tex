\documentclass[twoside]{brbrief}

\usepackage{lipsum}

\makeatletter

\makeatother


	\ZViii{elektronische Freimachungsvermerke}
	\ZVii{Vorausfügung}
	\ZVi{z.\,B.\ Einschreiben}
	\AZFirma{Firma}
	\AZAnrede{Frau/Herr}
	\AZName{Name des Addressaten}
	\AZStrasse{Zielstraße 1}
	\AZPLZOrt{12345 Zielort}
	\AZLand{Deutschland}
	\Betreff{Betreff}
	%\Briefkopf{}
	
\begin{document}
	\maketitle{}
	\Anrede{Liebe(r) Addressat(in)}
	
	\lipsum[1-2]
	
	Zu tun:
	\begin{itemize}
		\item Strich in Adressfeld genau so lang, wie längster Eintrag
		\item Dokumentation für jede Option
		\item Zusatzinfos unter Absenderinfos
		\item Location Absnederinfos configurable
		\item Briefpapier (Hintergrundbild/PDF) optional
	\end{itemize}
	
	\Gruss{Dein}
	
	% Post Scriptum
	%PS: Ich bin bis März nur telefonisch erreichbar.
	
	% Anlage(n)
	% Standardmäßig wird "Anlage(n)" eingefügt, dies kann überschrieben werden, hier mit "Anlage" oder "Anlagen"
	%\setkomavar*{enclseparator}{Anlage}
	%\setkomavar*{enclseparator}{Anlagen}
	%\encl{Kopie des Ausweises}
	
	% Weitere PDFs können automatisch angefügt werden, z.B. Ahnänge.
	%\includepdf[pages=-,openright]{pfad/zu/weiteren/pdfs/dokument.pdf}
	% Pfad ist relativ zu dieser tex-Datei. Mit .. ein Verzeichnis hoch.
	% Der pages-Parameter spezifiziert welche Seiten eingefügt werden.
	% Beispiele:
	% pages=-				alle Seiten
	% pages={1-4}			Seite 1-4
	% pages={1,4,5}			Seite 1, 4 und 5
	% pages={3,{},8-11,15}	Seite 3, leere Seite, Seite 8-11 und Seite 15
	% Der openright-Parameter startet die Anlagen auf ungerader (rechter) Seite, d.h. notfalls wird eine leere Seite
	% eingefügt. Im doppelseitigem Druck wird dadurch besser zwischen Brief und Anlage getrennt. Für einseitigen Druck
	% entfernen.
	
	
\end{document}
